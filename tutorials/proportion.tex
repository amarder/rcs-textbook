\chapter{How to estimate a model when the dependent variable is a proportion}

\section{QMLE Logit}

In a regression model, when the dependent variable is a proportion, a different approach than OLS is proposed by Papke and Wooldridge (1996)

\url{http://www.stata.com/support/faqs/stat/logit.html}

\url{http://www.ats.ucla.edu/stat/Stata/faq/proportion.htm}



I'll try to summarize what they suggest.

OLS in this case is inappropriate.  A proportion is bounded between 0 and 1, but the predicted values from an OLS regression can never be guaranteed to lie in the unit interval.

The transitional way of transforming the data by a logit function is not appropriate.  It can be formulated as 
\[ {\rm E} ({ \rm log} [y/(1-y)] | X ]= X \beta \]

This model has two drawbacks:

\begin{itemize}
\item The model cannot be true if $y$ takes 0 or 1.
\item Even if the model is well defined.  We cannot recover $ {\rm E} [ y | x] $ from it, which is ultimately what we want.
\end{itemize}


Papke and Wooldridge (1996) proposed the following model:

Their assumption is that for all $i$,

\begin{equation}
{\rm E} (y_i | x_i) = G(X_i \beta), 
\label{eq1}
\end{equation}
where $G()$ is a cumulative distribution function (cdf).  Most popular ones are logistic function and normal cdf.

Then the log-likelihood function is given by 
\begin{equation}
 l_i({\bf b}) = y_i \log [ G(X_i' {\bf b})] + (1- y_i) \log [ 1- G(X_i' {\bf b})], 
\label{eq2}
\end{equation}
which looks the same as our familiar logit log-likelihood function.  The difference is that in a logit model, $y_i$ is a binary variable; here $y_i$ can take any value between 0 and 1.  

The powerful result from Papke and Wooldridge (1996) is that they proved that under assumption \ref{eq1}, $\hat \bf b $ is consistent, regardless of the actual distribution of $y$ is.  This is why \ref{eq2} is called Quasi-MLE or QMLE since $y$ does not have to have a logistic distribution.  As long as the mean of $y$ is correctly specified the QMLE estimator is consistent.

Therefore, we should follow Stata's advice of using glm when we have a proportion as our dependent variable.

\section{beta distribution}

Another way to model a proportion is to assume beta distribution, that
is, $y \sim beta(\alpha, \beta)$.

Then model the mean of the distribution.

Specifically,

\[  { \rm log} [\mu /(1- \mu)] | X ]= X \eta\]

here $\mu = \frac{\alpha}{\alpha + \beta}$.  

In Stata, $betafit$ is the command which implements this.
