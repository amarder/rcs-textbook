\chapter{An introduction to Hausman-Taylor model}

\section{Hausman-Taylor model}

Random effects and fixed effects models are used widely in
econometrics for panel data.  Many economists tend to like
fixed-effect model better since it eliminates all the commonality
within an individual (or a firm, etc), therefore the unobserved individual
heterogeneity is controlled for.  However, in a fixed-effect model, any
covariates that are constant within individual cannot be included in
the estimation.  A random-effect model can have a time-invariate
variable in the regression; however, it assumes orthogonality between
error term and the individual effects, which is often not true.

Hausman-Taylor(1981) (HT) model uses a ``mixed'' structure to handle
this situation that we need to include a time-invariate variable and
model unobserved individual heterogeneity.  It's ``mixed'' in the
sense that it is between fixed effect and random effect; or a mixture
of both. 


Consider a standard panel data  model with time-invariate variables in it:
\[ y_{it}= X_{it}' \beta + Z_i \gamma + \mu_i +  \epsilon_{it}, \]

where $Z_i$ are cross-sectional time-invariant variables.  HT splits
the covariates into two sets: $X=[X_1;X_2]$ and $Z=[Z_1; Z_2]$ where
$X_1$ and $Z_1$ are exogenous and $X_2$ and $Z_2$ are endogenous, in
the sense that they are correlated with $\mu_i$ but not
$\epsilon_{it}$.  Then we have

\[ y_{it}= X_{1it}' \beta_1 +  X_{2it}' \beta_2 + Z_{1i} \gamma_1 + Z_{2i} \gamma_2 + \mu_i +  \epsilon_{it}, \]

First we do a ``within'' transformation, which is basically deduct all
the variables in the regression from its group mean (individual mean).
In that case, obviously $Z$'s would be removed.  Therefore we are left
with

\[ \tilde y_{it} = \tilde X_{1it} \beta_1 + \tilde X_{2it} \beta_2 +
\tilde \epsilon_{it}\]

where $\tilde y_{it}$ is the ``within'' transformed $y_{it}$, etc.

From this equation we can estimate the ``within'' estimator of
$\beta_1$ and $\beta_2$; call them $\hat \beta_{1w}$ and $\hat
\beta_{2w}$. 

Then we obtain the ``within'' residual:

\[ \tilde d_{it} = \tilde y_{it} - \tilde X_{1it} \hat \beta_{1w} -
\tilde X_{2it} \hat \beta_{2w} \]

The variance of the idiosyncratic error term, $\sigma^2_{\epsilon}$
can be estimated:
\[ \hat \sigma^2_{\epsilon} = \frac{RSS}{N-n} \]

where $RSS$ is the residual sum of squares from the within regression.

Now regress $\tilde d_{it}$ on $Z_1$ and $Z_2$, using $X_1$ and $Z_1$
as instruments.  We get $\hat \gamma_{1IV}$ and $\hat \gamma_{2IV}$,
which are consistent estimates of $\gamma_1$ and $\gamma_2$.

\[ \gamma_{IV}=(Z' P_A Z)^{-1} Z' P_A \hat d \]
where $P_A = A(A'A)^{-1}A' $ and $A= [X_1, Z_1]$ is a set of instruments.

With  $\hat \gamma_{1IV}$ ,  $\hat \gamma_{2IV}$ and $\hat \sigma_{\epsilon}^2$, we can estimate
$\hat \sigma_{\mu}^2$ (procedure of doing this omitted).

Then define $\hat \theta_i$ as

\[\hat \theta_i = 1 - (\frac{\hat \sigma_{\epsilon}^2}{\hat
  \sigma_{\epsilon}^2 + T_i \hat \sigma_{\mu}^2})^{1/2}\]

A random effect transformation can be done on each of the variables:

\[ w_{it}^* = w_{it} - \hat \theta_i \bar w_i \]

where $\bar w_i$ is the within-panel mean.  That is, each of $y$, $X$
and $Z$ are transformed in this way.  We have now 

\[ y_{it}^* = X_{it}^* \beta + Z_{it}^* \gamma + \epsilon_{it}^* \]


Then the HT estimator can be obtained by IV regression of $y_{it}^*$
on $X_{it}^*$ and $Z_{it}^*$, with instruments $\tilde X_{it}$, $\bar
X_{1i}$ and $Z_{1i}$.  

\section{Implemented in Stata}

In Stata, HT model is implemented as xthtaylor for version 10.1.
